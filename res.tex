\include{/Users/pluttan/Documents/forMyDocs/preamb.tex} % <!-- ~preamb -->
 % <!-- ~preambMisha -->
% //TODO <!-- ~whodo pluttan, fixii -->
\begin{document} % <!-- ~bd -->
1 %<!-- *print(1) -->
\LaTeX %<!-- ?\LaTeX -->
%<!-- ! --><!-- ^ --><!-- & -->
\section{\textbf{Заголовки}} % ## **Заголовки**

\dotitle{name}{subject} % # name : subject
\section{section} % ## section
\subsection{subsection} % ### subsection
\subsubsection{subsubsection} % #### subsubsection
\paragraph{paragraph} % ##### paragraph
\subparagraph{subparagraph} % ###### subparagraph
\section{\textbf{Акцент}} % ## **Акцент**
\textit{Курсив}, \textbf{Жирный}, \sout{Зачеркнутый} % *Курсив*, **Жирный**, ~~Зачеркнутый~~

\textit{Курсив}, \textbf{Жирный} % _Курсив_, __Жирный__

\begin{myquote}
Цитаты% >Цитаты 
\\% >
В 2 абзаца% >В 2 абзаца
\end{myquote}

\section{\textbf{Списки}} % ## **Списки**
\begin{enumerate}
\item Первый элемент нумерованного списка % 1. Первый элемент нумерованного списка
    \begin{enumerate}
        \item Первый элемент в первом элементе % 1. Первый элемент в первом элементе
    \end{enumerate}
\item Второй элемент нумерованного  списка % 2. Второй элемент нумерованного списка
\end{enumerate}

\begin{itemize}
    \item Первый элемент ненумерованного списка % * Первый элемент ненумерованного списка
        \begin{enumerate}
            \item Первый элемент в первом элементе % 1. Первый элемент в первом элементе
        \end{enumerate}
    \item Второй элемент ненумерованного  списка % * Второй элемент ненумерованного списка
\end{itemize}

\begin{itemize}
    \item Первый элемент ненумерованного списка % - Первый элемент ненумерованного списка
        \begin{itemize}
            \item Первый элемент в первом элементе % * Первый элемент в первом элементе
        \end{itemize}
    \item Второй элемент ненумерованного  списка % - Второй элемент ненумерованного списка
\end{itemize}

\begin{itemize}
    \item Первый элемент ненумерованного списка % + Первый элемент ненумерованного списка
    Просто текст
    \item Второй элемент ненумерованного  списка % + Второй элемент ненумерованного списка
\end{itemize}

\section{\textbf{Код}} % ## **Код**

Это код : \texttt{for i in range(1,100):print(i)} % Это код : `for i in range(1,100):print(i)`

% ```
% for i in range(1,100):print(i)
% ```

\begin{lstlisting}                  
for i in range(1,100):print(i)
\end{lstlisting}                    

%   for i in range(1,100):print(i)
\begin{lstlisting}
for i in range(1,100):print(i) 
\end{lstlisting}  

% ```python
% for i in range(1,100):print(i)
% ```

\begin{lstlisting}[language=python]  
for i in range(1,100):print(i)       
\end{lstlisting}                    

\section{\textbf{Изображения}} % ## **Изображения**

% //TODO %![100](https://liwli.ru/upload/iblock/460/0hR6uB5aWLz2O_XinqatMtzSPqzxbgiw.jpeg)

\section{\textbf{Ссылки}} % ## **Ссылки**

\href{https://vk.com/pluttan}

\section{\textbf{Символы}} % ## **Символы**
%\\ \` \* \_ \{} \[] \() \# \+ \- \. \! \|
$\backslash$ \` \* \_ \{\} [] () \# + - . ! |

\section{\textbf{Таблица}} % ## **Таблица**

\begin{center}
    \begin{tabular}{ l c r }
    2 & 2 & 2 \\        % | 2    |   2   |    2 |
                        % | :--- | :---: | ---: |
     1 & 2 & 3 \\       % | 1    |   2   |    3 |
     144 & 355 & 444 \\ % | 144  |  355  |  444 |
    \end{tabular}
\end{center}

\section{\textbf{Синтаксис}\LaTeX a} % ## **Синтаксис $\LaTeX$a**

Пусть функция $f(x)$ определена...
$$X = A^{-1}B \Rightarrow \begin{pmatrix}x_1\\x_2\\\vdots\\x_n\end{pmatrix} = \frac{1}{\Delta}
\begin{pmatrix}
    A_{11}&A_{21}&\ldots&A_{n1}\\
    A_{12}&A_{22}&\ldots&A_{n2}\\
    \vdots&\vdots&\ddots&\vdots\\
    A_{1n}&A_{2n}&\ldots&A_{nn}\\
\end{pmatrix}
\begin{pmatrix}b_1\\b_2\\\vdots\\b_n\end{pmatrix}
$$

\section{\textbf{Сноски (без предпросмотра)}}               % ## **Сноски (без предпросмотра)**
word\footnote{word - define word}                           % word <!-- ~note define word -->
three big word\footnote{three big word - define three word} % three big word <!-- ~note 3:define three word-->

\section{\textbf{Определения(без предпосмотра)}} % ## **Определения(без предпосмотра)**

\textbf{Apple}\\
\hspace{20pt}Pomaceous fruit of plants of the genus Malus in the family Rosaceae.

\end{document} % <!-- ~ed -->