
\include{/Users/pluttan/Documents/forMyDocs/preamb.tex}
\renewcommand{\copyright}{pluttan, fixii}
\begin{document}    
    
\LaTeX{} 
1

\section{\textbf{Заголовки}}
\dotitle{name}{subject}
\section{section}
\subsection{subsection}
\subsubsection{subsubsection}
\paragraph{paragraph}
\subparagraph{subparagraph}
\section{\textbf{Акцент}}

\textit{Курсив}, \textbf{Жирный}, \sout{Зачеркнутый}

\textit{Курсив}, \textbf{Жирный}
\begin{mainQuote}

Цитаты

В 2 абзаца
\end{mainQuote}
\section{\textbf{Списки}}
\begin{enumerate}

\item Первый элемент нумерованного списка
\begin{enumerate}

\item Первый элемент в первом элементе

\end{enumerate}
\item Второй элемент нумерованного списка

\end{enumerate}
\begin{itemize}

\item Первый элемент ненумерованного списка
\begin{enumerate}

\item Первый элемент в первом элементе

\end{enumerate}
\item Второй элемент ненумерованного списка
\item Первый элемент ненумерованного списка
\begin{itemize}

\item Первый элемент в первом элементе

\end{itemize}
\item Второй элемент ненумерованного списка
\item Первый элемент ненумерованного списка

Просто текст
\item Второй элемент ненумерованного списка

\end{itemize}
\section{\textbf{Код}}

Это код : \texttt{for i in range(1,100):print(i)}
\begin{lstlisting}
for i in range(1,100):print(i)
\end{lstlisting}
\begin{lstlisting}
for i in range(1,100):print(i)
\end{lstlisting}
\begin{lstlisting}[language = python]
for i in range(1,100):print(i)
\end{lstlisting}
\section{\textbf{Изображения}}


\image{img/000original.jpg}{120}

\section{\textbf{Ссылки}}

\hrf{https://vk.com/pluttan}{https://vk.com/pluttan}

\hrf{https://vk.com/pluttan}{Мой ВК}
\section{\textbf{Символы}}

` * \_ {} [] () \# + - . ! | > <
\section{\textbf{Таблица}}
\begin{center}
\begin{tabular}{|c|c|c|}
\hline



\textbf{2}&\textbf{2}&\textbf{2}\\
\hline



1&2&3\\

144&355&444\\

\hline
\end{tabular}
\end{center}
\section{\textbf{Синтаксис LaTeXa}}

Пусть функция $f(x)$ определена… $$X = A^{-1}B \Rightarrow \begin{pmatrix}x_1\\x_2\\\vdots\\x_n\end{pmatrix} = \frac{1}{\Delta}
\begin{pmatrix}
    A_{11}&A_{21}&\ldots&A_{n1}\\
    A_{12}&A_{22}&\ldots&A_{n2}\\
    \vdots&\vdots&\ddots&\vdots\\
    A_{1n}&A_{2n}&\ldots&A_{nn}\\
\end{pmatrix}
\begin{pmatrix}b_1\\b_2\\\vdots\\b_n\end{pmatrix}
$$
\section{\textbf{Сноски (без предпросмотра)}}

word \footnote{define word}
\section{\textbf{Определения(без предпосмотра)}}


\textbf{Apple}\\
\hspace{20pt}Pomaceous fruit of plants of the genus Malus in the family Rosaceae.



\end{document}
    